\documentclass[12pt]{article}
\usepackage{ctex}
\usepackage{graphicx}
\usepackage{caption}
\usepackage{tabularx}
\usepackage{geometry}

\geometry{a4paper,scale=0.75}


\title{电力电子技术课程设计}
\author{许文晋}
\begin{document}
\tableofcontents
\pagebreak
\counterwithin{figure}{section}
\counterwithin{equation}{section}

\section{选题背景}
自动化技术有着悠久的发展历史和广泛的应用。自动化是科学知识和使用机器的经验相结合的结果,以实现高生产力、高质量和生产任务的自动化。自动化技术广泛的应用在现代生产线上。

20世纪70年代,电子技术、传感器技术、控制技术的快速发展推动了自动化技术的发展。特别是计算机的普及使自动化技术有一个质的变化,计算机的发展推动了机器人技术的进步,自动化技术在各个行业的应用越来越广泛。自动化技术的应用提高了劳动效率,提升了产品质量,改善了工作环境。

如今的现代生产线是由机械技术、微电子技术、电工电子技术、传感与检测技术、接口技术、信息转换技术、网络通信技术等技术有机地联系在一起,集成到生产设备中;而系统是指生产线的传感与检测、传输与处理、控制、执行与驱动机制在微处理器单元的控制下协调有序地工作,有机地结合在一起。



\end{document}